\section{RELATED WORK}
This work is related to private and subtle interaction, and finger-worn devices, and magnetic tracking.

\subsection{Private and subtle interaction}
%mulscle\cite{Saponas:2009}, nenya\cite{Ashbrook:2011}, iRing\cite{Ogata:2012}, pinstripe\cite{Karrer:2010}
%footSense\cite{Scott:2010}, whack gestures\cite{Hudson:2010}
%pocket touch\cite{Saponas:2011}, ShoeSense\cite{Bailly:2012}, 
Several subtle interaction technique had been proposed. 
Costanza et al.  \cite{Costanza:2007} had used electromyography for sensing subtle motionless gestures. 
Saponas et al. \cite{Saponas:2009} used the similar technique for sensing different finger gestures.  
Scott et al. \cite{Scott:2010} proposed the idea of using mobile devices located in user's pocket to sense the foot gestures.
PingStripe \cite{Karrer:2010} allows users to perfrom subtle interactions by pinching and rolling a piece of their cloths.
Other works \cite{Ashbrook:2011, Ogata:2012} used ring as the subtle input device.
Still, these works can only support limited gesture input. 
FingerPad, on the other hand,  is functionaly equivient to a the touchpad, which implies that providing more dimensions for the input space.

\subsection{Finger-worn input devices}
%fingerRing \cite{Fukumoto:1994}, eyeRing\cite{Nanayakkara:2012}, magicFinger\cite{Yang:2012}, ubiFinger\cite{4030901}, irRing\cite{Roth:2010}
%such as Nenya[N] provide 1D input by rotating the ring on the finger, and iRing [R] additionally allows tapping input through the ring. 
%Disappearing device[D] allows marking menus through one-pixel motion scanner embedded in the ring device.
Several works using the finger-worn devices for sensing gestures had been proposed. 
FingerRing \cite{Fukumoto:1994} placed accelerometers on every finger of user's hand for sensing different chord gestures.     
UbiFinger \cite{4030901} allows users to control house appliances using finger gestures by placing IR transmitter and bending/touch sensors on the index finger in combination with the accelerometer on the wrist.
MagicFinger \cite{Yang:2012} extended the dimensions of touching gestures by mounting camera on the finger.
Since FingerPad using finger pinch gestures as the input, it increases both privacy and subtlety in comparrison to these works. 

%\subsection{Always-available interaction}
%Tracking the high degrees-of-freedom of human hands as inputs, Digits, a wrist-worn mobile hand tracking device, allows users to use full hand gestures on the move. This device, nevertheless, requires high power cameras and suffers from occlusion problem.

%\subsection{Eye-free interaction}

\subsection{Magnetic tracking}
%\neyna, Abracadabra, GaussSense ... 
Maginetic tracking had been used to sense the gestures in a remote distance. 
For instance, Han et al. \cite{4421009} tracked finger-mounted magnet for handwriting input. 
Similarly, Abracadabra \cite{Harrison:2009} used finger-mounted magnet to control the watch device.
Nenya \cite{Ashbrook:2011}, on the other hand, used the magnet mounted in the finger ring to control the device.
Liang et al. \cite{Liang:2012} used magnetic mounted in the stylus and the hull sensor array for enabling input on arbitrary surface.
In comparrison to these works, FingerPad provides more private and subtle input by mounting the hall sensors and magnet on the fingertips.
 

%\subsection{One-handed interaction}
%Others further remove the travel distance by allowing hand gestures. GestureWrist reads gestural input through a capacitive-sensing watch-bend. Digits allows users to use full hand gestures on the move, using a wrist-worn depth sensors. Gesture input, nevertheless, requires to memorize functional mappings. 
